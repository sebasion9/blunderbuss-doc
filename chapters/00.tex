\subsubsection*{Tytuł pracy} 
\Title

\subsubsection*{Streszczenie}  
Celem pracy jest zaprojektowanie i implementacja kompilatora języka programowania wzorowanego na języku C, w szczególności w zakresie składni oraz semantyki, który udostępnia programiście jawną kontrolę nad wykorzystaniem techniki optymalizacyjnej memoizacji. Mechanizm ten umożliwia zapamiętywania wyników deterministyczych i kosztownych obliczeniowo funkcji już na etapie analizy semantycznej oraz generacji kodu asemblerowego. Zastosowanie memoizacji pozwala na optymalizację czasu wykonywania programu wynikowego poprzez wymianę kosztu obliczeniowego na zwiększone zużycie pamięci.
\subsubsection*{Słowa kluczowe} 
gramatyka formalna, kompilator, assembler, język programowania

\subsubsection*{Thesis title} 
\begin{otherlanguage}{british}
\TitleAlt
\end{otherlanguage}

\subsubsection*{Abstract} 
\begin{otherlanguage}{british}
This thesis aims to design and implement a compiler for programming language inspired by C, with a focus on syntax and semantics, which provides the programmer with explicit control over the use of the opimization technique of memoization. This mechanism enables caching the results of deterministic and computationally expensive functions during semantic analysis and assembly code generation. The use of memoization allows for the optimization of the execution time of the resulting program by trading computational cost for increased memory usage (space-time trade-off).
\end{otherlanguage}
\subsubsection*{Key words}  
\begin{otherlanguage}{british}
formal grammar, compiler, assembler, programming language
\end{otherlanguage}

